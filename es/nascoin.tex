\section{Criptodivisa Nebulas (NAS)}
\label{sec:nascoin}

La red Nebulas posee su propia criptodivisa, cuya denominación es NAS. Ésta juega dos roles en la red: primero, como su criptodivisa original, proporciona liquidez de activos entre los usuarios, y funciona como el medio para el cobro de los incentivos para contadores (PoD) y desarrolladores (DIP); segundo, se utilizará esta criptodivisa para el cálculo de las comisiones por ejecución de contratos inteligentes.

La unidad mínima de NAS es $10^{-18}$ NAS.

Originalmente, NAS se distribuyó como un \textit{token} ERC20 en la plataforma Ethereum, con un total máximo agregado de $X = 10^8.$ Los modos de emisión son los siguientes:

\begin{enumerate}
	\item \textbf{Construcción comunitaria:}
	Bajo la dirección del equipo auspiciante de Nebulas, se utilizarán $80\%X$ \textit{tokens} para la construcción de la comunidad de Nebulas, incluyendo la incubación ecológica y los incentivos de la \dapp \textit{Nebulas community blockchain}, la construcción de la comunidad de desarrolladores, la cooperación de industrias y negocios, marketing y promoción, investigación académica, inversión en educación, leyes y regulaciones, e inversiones en comunidades y organizaciones. Específicamente, se venderán $5\%X$ NAS a distintos inversionistas de impacto comunitario, $5\%X$ NAS como fondo de desarrollo comunitario de Nebulas y $70\%X$ NAS se mantendrán como reserva.

	\item \textbf{Incentivos para patrocinadores y equipos de desarrollo:}
	A lo largo del desarrollo de Nebulas, el patrocinador y los equipos de desarrollo realizarán contribuciones continuas de materiales y recursos humanos en aspectos relacionados a la estructura organizativa del proyecto, investigación, desarrollo tecnológico y operación ecológica. En cuanto a la asignación de \textit{tokens}, se reservan $20\%X$ NAS como incentivos para el equipo. Esta parte se bloqueará inicialmente, y se desbloqueará un año después de la finalización de la primera venta de NAS a la comunidad, distribuyéndose gradualmente a los equipos de patrocinadores y de desarrollo a lo largo de tres años.

\end{enumerate}

Luego del lanzamiento oficial de la red Nebulas, cualquier usuario con tokens NAS ERC20 podrá cambiarlos por la criptodivisa nativa de la red Nebulas (también llamada NAS) usando para ello las credenciales relacionadas. A medida que la red Nebulas evolucione, NAS crecerá de la siguiente manera:

\begin{enumerate}
	\item \textbf{Incentivos PoD:} $3\%X$ NAS de incentivos por año para los \textit{contadores};

	\item \textbf{Incentivos DIP:} $1\%X$ NAS de incentivos por año para los desarrolladores destacados de contratos inteligentes.

\end{enumerate}