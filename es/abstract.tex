\abstract{

Tanto Bitcoin como Ethereum introdujeron satisfactoriamente las ideas de “sistema de dinero electrónico p2p” y “contratos inteligentes” al mundo del blockchain. La industria evoluciona rápidamente, con más aplicaciones emergentes y requerimientos; sin embargo, para las tecnologías blockchain actuales, hemos visto que existen tres desafíos centrales: lograr el establecimiento de una medida de valuación para las aplicaciones en el blockchain, implantar la habilidad de la auto-evolución incorporando los futuros avances en los bloques, y establecer un ecosistema sano a largo plazo, disponible para todos los participantes del mundo blockchain.

  Nebulas apunta a darles curso a estos desafíos. Este libro blanco explica la ideología detrás del diseño técnico y los principios del \textit{framework} de Nebulas; este último incluye:

\begin{itemize}
  \item \textbf{Nebulas Rank (NR)} (\refsec{sec:rank}), un nuevo sistema para la valuación de aplicaciones en el blockchain.
  Nebulas Rank mide la valuación \footnote{N. del T.: \textit{Ranking}, en inglés} de las aplicaciones teniendo en cuenta la liquidez y la propagación tanto de las direcciones como de los contratos inteligentes utilizados en las {\dapp}s de la plataforma Nebulas, en una forma veraz, computable y determinista. Con nuestro nuevo sistema de valuación seremos capaces de ver asomar más {\dapp}s con un uso real en la plataforma Nebulas.


	\item \textbf{Nebulas Force (NF)} (\refsec{sec:nebulasforce}), cuyo fin es el de ofrecer la actualización de los protocolos centrales y de los contratos inteligentes de forma directa en los \textit{main chains}, o cadenas principales. Esto le brindará a Nebulas la capacidad de auto-evolucionar y de incorporar tecnología de punta en sus blockchains, una vez que estén listas para su uso en el mundo real. Con Nebulas Force, los desarrolladores podrán construir aplicaciones sofisticadas en forma rápida, and these applications can dynamically adapt to market changes or community feedback.


	\item \textbf{Developer Incentive Protocol (DIP)} (\refsec{sec:dip}), está pensado para desarrollar nuestro ecosistema blockchain mediante la otorgación de tokens Nebulas, a modo de incentivo, a los mejores desarrolladores de nuestra plataforma, quienes serán determinados mediante nuestro sistema Nebulas Rank. Esto ayudará a recompensar las mejores aplicaciones, y a incentivar a sus desarrolladores a crear más valor, tanto para sí mismos como para Nebulas.


  \item \textbf{Algoritmo de consenso Prueba de Devoción (\textit{Proof of Devotion}, o PoD)} (\refsec{sec:pod}). Para crear un ecosistema sano, Nebulas propone tres puntos clave para su algoritmo de consenso: celeridad, irreversibilidad, y equidad.

  \item \textbf{Motor de búsquedas para {\dapp}s}~(\refsec{sec:search}). Nebulas está construyendo un nuevo tipo de motor de búsqueda para {\dapp}s, basado en el sistema Nebulas Rank. Al usar este motor, los usuarios serán capaces de encontrar las {\dapp}s más útiles y ajustadas a sus necesidades.

\end{itemize}
}