\section{星云链账号地址设计}
星云链的账号地址被精心设计为以``n''开头的形式,以表示其是一个星云(Nebulas)的地址。

\subsection{用户账户地址}
类似于比特币和以太坊,星云链的帐号也是由椭圆曲线算法作为其基础加密算法。
用户的私钥是通过加密用户的口令得到的,进一步的,由该私钥生成相应的公钥。


此外,为了防止用户输错了几个字符而把星云币误发给其它用户,星云链的地址包含了校验码。

星云链中的用户地址计算方式如下:

\begin{verbatim}
content = ripemd160(sha3_256(public key))
checksum = sha3_256(0x19<<(21*8) + 0x57<<(20*8) + content)[0:4]
address = base58(0x19<<(25*8) + 0x57<<(24*8) + content<<(4*8) + checksum)
\end{verbatim}
其中0x19为一个字节的填充,0x57是用户账户地址对应的类型。\texttt{content}的长度为20个字节,\texttt{checksum}的长度为4个字节,\texttt{address}的长度为35个字符。

在使用base58编码之后,一个典型的地址如下:n1TV3sU6jyzR4rJ1D7jCAmtVGSntJagXZHC,即以字母n开头。



\subsection{智能合约地址}
在星云链中,对智能合约地址的计算方式略微区别于普通的用户账户地址,对于智能合约的地址而言,不需要额外的口令,其计算方式如下:

\begin{verbatim}
content = ripemd160(sha3_256(tx.from, tx.nonce))
checksum = sha3_256(0x19<<(21*8) + 0x58<<(20*8) + content)[0:4]
address = base58(0x19<<(25*8) + 0x58<<(24*8) + content<<(4*8) + checksum)
\end{verbatim}
\noindent
可以看到,这类似于普通用户帐户地址的计算方式,所不同的是,智能合约地址对应的类型为0x58。同样的,一个典型的智能合约地址如下:
n1sLnoc7j57YfzAVP8tJ3yK5a2i56QrTDdK。

