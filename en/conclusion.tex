\section{Conclusion}
\label{sec:conclusion}

\subsubsection*{What We Hold}

From a highly abstract perspective, blockchain is \textbf{the right confirmation of data in a decentralized way}, and tokens function as \textbf{the carrier of right confirmation value}. The Internet solves the communication of data, while blockchain further solves the right confirmation of data. For the first time ever, blockchain translates public data into private data which will no longer be analyzed and utilized arbitrarily by large enterprises such as Google, Amazon and Facebook.

%区块链从一种高度抽象的角度来看是一种\textbf{用去中心化的方式对于数据的确权},代币本身是对于\textbf{确权价值的载体}。互联网解决了数据的通讯问题,而区块链则在互联网上层进一步解决了数据的确权问题。区块链前所未有第一次让大家的数据真正变成自己的数据,而不再被BAT等大公司任意分析并使用。

The essence of blockchains represented by public blockchains is \textbf{``Community + Token + Toolsdi"}. Community is essentially a from-bottom-to-top ecosystem adhering to the idea of openness, source opening, sharing and non-profitability, which is completely different from existing from-top-to-bottom business ecosystems. Token is the carrier of right confirmation value. There will be more scenarios in the future than those used only for attributes of virtual currency and electronic cash. Tools simply refer to the technological implementation of blockchain application scenarios. Without the combination with the aforementioned two factors, tools alone \textbf{cannot fully reflect the charm of blockchain systems}.

%以公有链为代表的区块链本质精神是:\textbf{社群+代币+工具}。社群本质上是自下而上,秉承的是开放、开源、共享、非盈利的理念,和现有自上而下的商业生态有着根本的不同。代币即是对于确权价值的载体,未来会有更多的应用场景,而远非是仅仅面向虚拟货币,电子现金的属性。工具仅仅是对于区块链应用场景具体技术实施,如缺少前两者的结合,其单独来看\textbf{并不能完全体现区块链系统的魅力所在}。

The blockchain system represented by public blockchains is the future of blockchains, as its \textbf{``untrusted"} and \textbf{``unprivileged"} features are the actual value of blockchain systems. On the contrary, most consortium blockchains/enterprise blockchains are \textbf{``trust-based"} and \textbf{``privilege-based"}, which means they cannot break existing patterns and are considered as improved innovations. While public blockchain systems overturn existing cooperation relations and are considered \textbf{disruptive innovations}, reflecting the maximum value of blockchains.

%以公有链为代表的区块链体系才是区块链的未来,因为其本身所具有的“\textbf{非信任}”、“\textbf{无特权}”的基本特性才是区块链系统真正的价值所在。恰恰相反,作为联盟链/企业链大多具有“\textbf{基于信任}”和“\textbf{基于特权}”的属性,不能突破既有的范式,属于改良式创新。而公有链系统颠覆了既有的协作关系,属于\textbf{颠覆式创新},是区块链价值最大化的真正体现。

\subsubsection*{What We Are Committed To}

Be as the first blockchain search engine around the world, Nebulas is committed to \textbf{exploring hidden dimensions of blockchain value} and building value-based blockchain operating systems, search engines and other related extensive apps.

%作为全球首个区块链搜索引擎,星云链致力于\textbf{发掘区块链世界价值新维度},打造基于价值尺度的区块链操作系统、搜索引擎及其他相关扩展应用。

With this commitment, we put forward \textbf{Nebulas Rank} to set up the measure of value of the blockchain world, design \textbf{Nebulas Force} to empower the self-evolution of blockchains, develop \textbf{Developer Incentive Protocol} and \textbf{Proof of Devotion} to motivate the upgrade of blockchain value, and construct \textbf{Nebulas Search Engine} to help users explore other dimensions of blockchain value.

%基于此,我们提出Nebulas Rank星云指数来构建区块链世界的价值尺度,设计Nebulas Force星云原力来赋予区块链自我进化的能力,推出Developer Incentive Protocol开发者激励计划和Proof of Devotion贡献度共识证明来激励区块链的价值升级,打造Nebulas Search Engine区块链搜索引擎来帮助用户发现区块链上沉淀的多维度价值。

\subsubsection*{What We Believe}

The ongoing scientific and technological evolution will lead us to a better life with a higher level of \textbf{freedom and equality}. As one of the major technologies, blockchain will gradually give full play to its advantages. Being part of this evolution is our greatest happiness and accomplishment.

%正在发生科技浪潮会带领人们抵达更为\textbf{自由、平等、和美好}的生活。区块链作为其中重要的技术之一,会愈来愈散发出其特有的光彩和能量。能参与并投身其中,是我们最大的快乐和成就。

Similar to the Internet, blockchains will also enter a phase of explosive users/apps. Blockchain technology will become the \textbf{base protocol} of the next-generation ``smart network", and the number of users will reach or even go beyond one billion in the next 5 to 10 years. Significant opportunities and challenges will both emerge in the next five years.

%类似互联网对于世界的改变,区块链也即将面对其用户/应用临界值爆发的阶段。区块链技术会是下一代“智能网络”的\textbf{基础协议},整体用户规模会在5-10年内达到或超过10亿。未来的5年内会面临重大的机遇与挑战!

Facing the tremendous ecosystem in the future, do not wonder what blockchains can do for you but what you can do for blockchains. Because \textbf{blockchains themselves are both life entities and economic entities}. We are glad to share these with all of you in the exploration of blockchain technologies.

%在未来的巨大生态面前,在当下,不要问区块链能为你做什么,而是要问你能为区块链做什么。因为,\textbf{区块链本身就是生命体,区块链本身就是经济体}!在区块链技术探索的道路上,与诸君共勉!
