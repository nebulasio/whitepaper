\abstract{
  %比特币和以太坊系统分别给区块链世界带来"去中心化现金"和"智能合约"技术。如今,区块链技术及产业已经取得了长足的发展和繁荣,各种应用场景、商业需求层出不穷。随之而来的,我们发现,已有的区块链技术已经不能满足日益增长的用户需求,总的来说,区块链技术面临着价值尺度缺失、自我进化及生态建设三方面的挑战。
Bitcoin and Ethereum respectively contributed ``decentralized currency" and ``smart contract" technologies to the blockchain world. Blockchain technology and industry have since seen considerable development, with a variety of application scenarios and business needs springing up. But gradually we find that existing blockchain technologies are increasingly falling short of the growing needs of users, and three challenges stand out: lack of measure of value, self-evolution and ecosystem construction.

  %本文介绍了星云链的技术架构设计,意图构建一个能够量化价值尺度、具备自进化能力,并能促进区块链生态建设的区块链系统,主要内容包括:
  This paper introduces the design of Nebulas technical framework, which aims to build a blockchain system that is able to quantify measure of value, self-evolve and promote the construction of blockchain ecosystem. The framework mainly includes:

\begin{itemize}
  \item \textbf{Nebulas Rank} (NR) ~(\refsec{sec:rank}), a measure of value aiming for a universally credible, computable and replicable measurement future for each account. It is foreseeable that with ongoing exploration of in-depth value based on the liquidity and propagation of individual accounts on the chain, there will emerge more diverse applications on the Nebulas platform.
  %\textbf{定义价值尺度的星云指数Nebulas Rank(NR)}~(\refsec{sec:rank}),通过综合考虑链中各个账户的流动性及传播性,NR试图为每个账户建立一个可信、可计算及可复现的普适价值尺度刻画。可以预见,在NR之上,通过挖掘更大纵深的价值,星云链的平台上将会涌现更多、更丰富的应用。

	\item \textbf{Nebulas Force}~(\refsec{sec:nebulasforce}) which drives the upgrade of protocol code and smart contracts on the chain. It enables the self-evolution of the Nebulas system and its applications to dynamically adapt to community or market changes, allowing the rapid development and resilience building of Nebulas and its applications. Developers can also build more diverse applications on Nebulas and achieve fast iteration.
	%\textbf{支持核心协议和智能合约链上升级的星云原力Nebulas Force(NF)}~(\refsec{sec:nebulasforce}),帮助星云链自身及其上的应用实现自我进化,动态适应社区或市场变化,从而使得星云链及应用将会有更快的发展速度和更大的生存潜力。开发者亦能够通过星云链构建更丰富的应用,并进行快速迭代。

	\item \textbf{DIP}, or Developer Incentive Protocol~(\refsec{sec:dip}), designed to better build the blockchain ecosystem. With the incentive of NAS, Nebulas aims to draw top developers to work for a diverse and value-added system.
    %\textbf{开发者激励协议Developer Incentive Protocol(DIP)}~(\refsec{sec:dip}),为了更好地建立区块链应用生态环境,星云链将通过星云币(NAS)来激励为⽣态助⼒的优秀应用开发者,促进星云链更加丰富多元的价值沉淀。

  \item \textbf{Proof of Devotion (PoD) Consensus Algorithm}~(\refsec{sec:pod}). To build a healthy and unconstrained ecosystem, Nebulas proposes three metrics of consensus algorithm, namely speed, irreversibility and fairness. By integrating the advantages of PoS and PoI, and based on the measure of value of the Nebulas system, PoD takes the lead in adding a third metric of fairness in addition to the speed and irreversibility.
  %\textbf{贡献度证明共识算法Proof of Devotion(PoD)}~(\refsec{sec:pod}),从星云链生态健康自由发展出发,星云链提出了共识算法的三个重要指标,即快速、不可逆和公平性,PoD通过融合PoS和PoI的优势,结合星云链中的价值尺度,在保证快速和不可逆的前提下,率先加入了公平性的考量。

  \item \textbf{Decentralized search engine}~(\refsec{sec:search}), a search engine for decentralized applications based on Nebulas’ measure of value. With this engine, users can sort out among the massive blockchain applications the ones that meet their expectations and needs.
    %\textbf{去中心化应用的搜索引擎}~(\refsec{sec:search}),基于我们所定义的价值尺度,星云链构建了一个针对去中心化应用的搜索引擎,帮助用户在海量区块链应用中,找到符合用户期望及应用场景的应用。

\end{itemize}
}
